% Options for packages loaded elsewhere
\PassOptionsToPackage{unicode}{hyperref}
\PassOptionsToPackage{hyphens}{url}
%
\documentclass[
]{book}
\usepackage{amsmath,amssymb}
\usepackage{iftex}
\ifPDFTeX
  \usepackage[T1]{fontenc}
  \usepackage[utf8]{inputenc}
  \usepackage{textcomp} % provide euro and other symbols
\else % if luatex or xetex
  \usepackage{unicode-math} % this also loads fontspec
  \defaultfontfeatures{Scale=MatchLowercase}
  \defaultfontfeatures[\rmfamily]{Ligatures=TeX,Scale=1}
\fi
\usepackage{lmodern}
\ifPDFTeX\else
  % xetex/luatex font selection
\fi
% Use upquote if available, for straight quotes in verbatim environments
\IfFileExists{upquote.sty}{\usepackage{upquote}}{}
\IfFileExists{microtype.sty}{% use microtype if available
  \usepackage[]{microtype}
  \UseMicrotypeSet[protrusion]{basicmath} % disable protrusion for tt fonts
}{}
\makeatletter
\@ifundefined{KOMAClassName}{% if non-KOMA class
  \IfFileExists{parskip.sty}{%
    \usepackage{parskip}
  }{% else
    \setlength{\parindent}{0pt}
    \setlength{\parskip}{6pt plus 2pt minus 1pt}}
}{% if KOMA class
  \KOMAoptions{parskip=half}}
\makeatother
\usepackage{xcolor}
\usepackage{longtable,booktabs,array}
\usepackage{calc} % for calculating minipage widths
% Correct order of tables after \paragraph or \subparagraph
\usepackage{etoolbox}
\makeatletter
\patchcmd\longtable{\par}{\if@noskipsec\mbox{}\fi\par}{}{}
\makeatother
% Allow footnotes in longtable head/foot
\IfFileExists{footnotehyper.sty}{\usepackage{footnotehyper}}{\usepackage{footnote}}
\makesavenoteenv{longtable}
\usepackage{graphicx}
\makeatletter
\def\maxwidth{\ifdim\Gin@nat@width>\linewidth\linewidth\else\Gin@nat@width\fi}
\def\maxheight{\ifdim\Gin@nat@height>\textheight\textheight\else\Gin@nat@height\fi}
\makeatother
% Scale images if necessary, so that they will not overflow the page
% margins by default, and it is still possible to overwrite the defaults
% using explicit options in \includegraphics[width, height, ...]{}
\setkeys{Gin}{width=\maxwidth,height=\maxheight,keepaspectratio}
% Set default figure placement to htbp
\makeatletter
\def\fps@figure{htbp}
\makeatother
\setlength{\emergencystretch}{3em} % prevent overfull lines
\providecommand{\tightlist}{%
  \setlength{\itemsep}{0pt}\setlength{\parskip}{0pt}}
\setcounter{secnumdepth}{5}
\usepackage{booktabs}
\ifLuaTeX
  \usepackage{selnolig}  % disable illegal ligatures
\fi
\usepackage[]{natbib}
\bibliographystyle{plainnat}
\IfFileExists{bookmark.sty}{\usepackage{bookmark}}{\usepackage{hyperref}}
\IfFileExists{xurl.sty}{\usepackage{xurl}}{} % add URL line breaks if available
\urlstyle{same}
\hypersetup{
  pdftitle={BIOSTAT 214},
  pdfauthor={Yilan Huang},
  hidelinks,
  pdfcreator={LaTeX via pandoc}}

\title{BIOSTAT 214}
\author{Yilan Huang}
\date{2023-10-06}

\begin{document}
\maketitle

{
\setcounter{tocdepth}{1}
\tableofcontents
}
\hypertarget{about}{%
\chapter{About}\label{about}}

For BIOSTAT 214 (Finite Population Sampling) assignments.

\hypertarget{hw1}{%
\chapter{HW1}\label{hw1}}

Show that the mean of sample means over all possible samples drawn from a finite population using SRSWOR is equal to the mean of the finite population.

\textbf{Proof:}

Let \(N\) = number of units in the population, with the population mean
\[
\bar{Y} = \frac{1}{N} \sum_{i=1}^N y_i. 
\]

Let \(n\) = size of a sample. When we draw the sample using SRSWOR, there are \(N \choose n\) distinct samples. The number of samples of size \(n\) that will contain a given unit in a population of size \(N\) is
\[
{N \choose n} \times \frac{n}{N} = {N-1 \choose n-1}.
\]

Let \(S = \{s: s = \{i_1 < i_2 < ... < i_n\}, (i_1, ..., i_n) \subseteq (1, ..., N)\}\). We have \(|S|\) = \(N \choose n\).

Let \(\bar{y_s}\) = sample mean of \(s \in S\). So,
\[
\bar{y_s} = \frac{1}{n} \sum_{i \in s} y_i.
\]

Then the mean of sample means over all possible samples is equal to
\begin{align}
\frac{1}{{N \choose n}} \sum_{s \in S} \bar{y_s} &= \frac{1}{{N \choose n}} \sum_{s \in S} (\frac{1}{n} \sum_{i \in s} y_i) \\
&= \frac{1}{{N \choose n}} \frac{1}{n} \sum_{s \in S} \sum_{i \in s} y_i \\
&= \frac{1}{{N \choose n}} \frac{1}{n} \sum_{s \in S} \sum_{s \ni i} y_i = (*) 
\end{align}

Since for a given \(i\), \(y_i\) will appear in exactly \({N-1 \choose n-1}\) samples,
\begin{align} 
(*) &= \frac{{N-1 \choose n-1}}{n{N \choose n}} \sum_{i=1}^N y_i \\
&= \frac{1}{N} \sum_{i=1}^N y_i = \bar{Y} 
\end{align}
which concludes the proof.

  \bibliography{book.bib,packages.bib}

\end{document}
